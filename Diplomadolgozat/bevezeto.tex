%----------------------------------------------------------------------------
\chapter{Bevezető}%\addcontentsline{toc}{chapter}{Bevezető}
%----------------------------------------------------------------------------

\lstdefinelanguage{JavaScript}{
  keywords={
    typeof, new, true, false, catch, function, return, null, catch, switch, var,
    if, in, while, do, else, case, break, export, import
  },
  keywordstyle=\color{blue}\bfseries,
  ndkeywords={
    class, extends, const, let, constructor, super, static
  },
  ndkeywordstyle=\color{purple}\bfseries,
  identifierstyle=\color{black},
  sensitive=false,
  comment=[l]{//},
  morecomment=[s]{/*}{*/},
  commentstyle=\color{green}\ttfamily,
  stringstyle=\color{red}\ttfamily,
  morestring=[b]',
  morestring=[b]",
  literate=
    *{0}{{\textcolor{blue}{0}}}{1}
    {1}{{\textcolor{blue}{1}}}{1}
    {2}{{\textcolor{blue}{2}}}{1}
    {3}{{\textcolor{blue}{3}}}{1}
    {4}{{\textcolor{blue}{4}}}{1}
    {5}{{\textcolor{blue}{5}}}{1}
    {6}{{\textcolor{blue}{6}}}{1}
    {7}{{\textcolor{blue}{7}}}{1}
    {8}{{\textcolor{blue}{8}}}{1}
    {9}{{\textcolor{blue}{9}}}{1}
    {\ }{{ }}{1},
  keywords=[2]{random, PBKDF2, toString, stringify, encrypt, parse, decrypt},
  keywordstyle=[2]\color{green}\bfseries
}

\section {Témaválasztás indoklása}

Napjainkban az online jegyvásárlás nagy előretörést ért el a technológia fejlődésével az egyre szélesebb körben történő bankkártyás internetes vásárlások következtében. Egy online platformon keresztül az emberek ma már kényelmesen és gyorsan tudnak jegyeket vásárolni különféle eseményekre, mivel percek alatt el tudják végezni a világ bármely pontjából a nap bármely időpontjában. Az online jegyvásárlás számos előnnyel jár, amelyeknek köszönhetően egyre népszerűbbé válik.

Az egyik legfontosabb előny az online jegyvásárlás esetén az, hogy a vásárlók egyszerűen és kényelmesen böngészhetnek és választhatnak a széles körű jegyválaszték közül. Az online platformok részletes információkat biztosítanak az eseményekről, így a potenciális vásárlók teljes körű tájékoztatást kapnak az eseményről, mielőtt eldöntenék, hogy vásárolnak-e jegyet. Tehát a vásárlók magabiztosan dönthetnek arról, hogy melyik eseményre szeretnének jegyet vásárolni, anélkül, hogy bármilyen korlátozásba ütköznének.

A technológiai fejlesztések, mint például a biztonságos online fizetési rendszerek (\ref{abra:Logok}), amelyek az EDI rendszerek alkomponensei, és az elektronikus jegyek, hozzájárultak az online jegyvásárlás népszerűségéhez. A vásárlók könnyedén és biztonságosan fizethetnek az online platformokon keresztül, és elektronikus jegyet kapnak, amelyet mobil eszközükön vagy nyomtatható formában mutathatnak fel az eseményen. Az elektronikus jegyek további előnye, hogy nehezen veszíthetőek el vagy semmisülhetnek meg, így a vásárlók biztonságban tudhatják jegyeiket. Itt fontos megemlíteni, hogy ezen jegyek tárolása és biztonságban tartása további adatbiztonsági kérdéseket vet fel, amellyel foglalkozunk a dolgozat keretében. A platformokon keresztül a vásárlók egyszerűen választhatják ki a kívánt eseményt, a megfelelő ülőhelyet vagy jegytípust, és azonnal megvásárolhatják a jegyüket néhány kattintással. Ez időt és energiát takarít meg a vásárlók számára, ami napjainkban egy lényeges tényező.

\begin{figure}[!h]
	\centering
	\includegraphics[scale=0.2]{images/logok}
	\caption{Online fizetési rendszerek}
	\label{abra:Logok}
\end{figure}
\pagebreak

További előnyeként egy ilyen platformnak megemlítendő, hogy a szervezők számára is jelentősen hatékonyabbá teszi a rendelések nyomon követését, statisztikák készítését, amelyeket felhasználhatnak értékesítési jelentések és a marketing javításához. Emellett az online jegyvásárlás lehetőséget nyújt a szervezőknek arra is, hogy célzottan reklámozzák az eseményüket, így nagyobb látogatottságot érhetnek el.

A jelenleg is működő hivatalos platform, ahol direkt módon juthatunk hozzá jegyekhez az FIA Formula-1-es versenyhétvégékre, az F1 Experiences. Itt gyorsan és kényelmesen vásárolhatjuk meg a kívánt jegyünket, amelyet a sikeres rendelés és kifizetés után emailben kapjuk meg PDF formátumban, amely tartalmazza a megvásárolt jegy(ek)et, egyedi azonosítókat, QR kódokat és a számlázási adatokat. Az eseményre érve, a belépő kapuknál, egy erre a célra kihelyezett okos eszközzel megtörténik a QR kód olvasása és hitelesítése. Pozitív eredmény esetén beléphetünk az esemény helyszínére.

A fent említett folyamat megengedi, hogy ezek az elektronikus jegyek átruházhatóak a tulajdonos által bárki számára. Ezzel persze önmagában nincs probléma, mivel ezt a szabadságot meg kell lehessen adni a felhasználóknak, hogy bizonyos esetekben más személy tudjon részt venni a vásárló helyett, így nem vész kárba a vásárlás. Ez a rendszer viszont teret ad egy olyan biztonsági kérdésnek, amelyet jelenleg a felhasználó felelősségére van bízva, miszerint ezt a kódot akár hetekkel, hónapokkal a használatuk előtt kapnak meg a felhasználók elektronikus levél formájában és azt bárki megszerezheti, akinek hozzáférése van a fiókhoz. Rosszabb esetekben, egy szándékos kibernetikai támadás esetén is eltulajdoníthatják és felhasználhatják. Ennek persze kisebb a valószínűsége, viszont ami egy aggasztó tény, hogy a felhasználók nagy része nem megfelelő módon kezeli az adatainak a biztonságos tárolását és számos esetben fellelhetőek olyan emberi hibák, amelyeket kihasználnak az adathalászok, hogy hozzáférjenek a megvásárolt jegyekhez és saját célokra használják fel, többnyire illegális módon kereskedjenek velük.

\pagebreak
Gyakran megtörténik, hogy egy felhasználó több oldalra is ugyanazokat a bejelentkezési adatokat adja meg a regisztráció során. Ez többségében a személyes email fiók felhasználó nevével és jelszavával megegyezik és ezt az adathalászok is figyelembe veszik. Egy másik sebezhetőség, hogy számos esetben egy fiókhoz több személy is hozzáfér, így már nem beszélhetünk biztonságos adattárolásról. Megemlítendő viszont, hogy az email szolgáltatók biztosítanak E2EE-t az elektronikus levelek küldésekor és fogadásakor.

Az F1 Ticket Manager webes alkalmazás célja az alapvető jegyvásárlási funkcionalitások biztosítása, valamint a teljes vásárlási folyamat biztonságosabbá tétele. Ennek megvalósítására számos technológiai fejlesztés és programozói technika létezik. Az alkalmazás fejlesztésénél beépítésre került PIN kódok használata, amely egy emelt szintű biztonságot nyújt a felhasználó számára, valamint egy titkosítási algoritmus, amely az eredeti adatok azonnali visszafejtését hivatott megnehezíteni. 