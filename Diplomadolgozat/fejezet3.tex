%----------------------------------------------------------------------------
\chapter{Rendszerspecifikáció} \label{fejezet3}
%----------------------------------------------------------------------------

\section {Rendszer követelmények}

A rendszer specifikációit négy nagy részre oszthatjuk fel, amelyek közé tartoznak a felhasználói felület-, a jegyvásárlás-, a felhasználók- és a biztonság specifikációja. Elsőként a felhasználói felület biztosítja a felhasználó és a rendszer közötti kommunikációt. A jegyvásárlás a rendszer központi funkcionalitása, ezáltal a legkomplexebb is. Az alkalmazás lehetővé teszi a több típusú felhasználók hozzáférését a különbőző funkcionalitásokhoz. Végül de nem utolsó sorban a rendszer a bizalmas adatok biztonságát is hivatott kiszolgálni. A specifikációk prezentálására a Visual Paradigm Online felület segítségével hoztam létre a szükséges diagramokat \cite{VPO}. A rendszer működését vizsgálni tudjuk a funkcionális- és nem funkcionális követelmény alapján.
\subsection {Funkcionális követelmények} \label{rendszerFunkcionális}

A funkcionális követelmények a rendszer funkcionalitásaira vonatkoznak, azaz meghatározzák, hogy milyen feladatokat és műveleteket kell elvégeznie a rendszernek, milyen funkciókat kell biztosítania a felhasználók számára.

\begin{itemize}
  	\item[\textbf{a,}] \textbf{Felhasználói szerepkörök:}

Az oldal böngészésének tekintetében érdemes különboző jogosultságokkal rendelkező szerepköröket kialakítani. Erre azért van szükség, mivel a rendszer több típusú felhasználó által látogatható. Ezek közül a legelső és egyben a legkevesebb funkcionalitással rendelkező az \textbf{Anonymous}, amelyek \textit{be nem jelentkezett} felhasználók, esetében ilyen funkcionális követelmények például a jegyek közötti böngészés lehetősége, a jegyek részleteinek megtekintése, valamint az email-küldés a \textit{Support} számára. Emellett fontos, hogy az Anonymous felhasználók be tudjanak jelentkezni az oldalra vagy új felhasználói fiókot tudjanak regisztrálni, és ez a bejelentkezési folyamat lehetőséget biztosít az oldalon regisztrált fiókok vagy akár más közösségi fiókok használatára is (\ref{abra:useCaseNA}).

\begin{figure}[!h]
	\centering
	\includegraphics[scale=0.7]{images/useCaseNA}
	\caption{Use case diagram - Nem bejelentkezett felhasználó}
	\label{abra:useCaseNA}
\end{figure}
\pagebreak

A \textbf{Regular} felhasználók esetében elvárt, hogy \textit{be legyenek jelentkezve} a rendszerbe. Lehetőségük van a jegyek kosárba helyezése, visszajelzések írása és jegyek értékelése. Emellett a felhasználók képesek megtekinteni a profiljukat, ahol lehetőségük van felhasználói név és profil kép változtatására, valamint a vásárlási előzményeik megtekintésére. Az előzmények részleteinél a felhasználóknak lehetőségük van az adott jegy PIN kódjának megváltoztatására, amennyiben szükséges. A vásárlási folyamat során a felhasználók a rendelési oldalra (\textit{Checkout}) jutnak el, ahol lehetőségük van a listaelemek mennyiségének módosítására vagy azok törlésére. Itt történik meg a fizetés és a rendelés leadása (\ref{abra:useCaseA}).

\begin{figure}[!h]
	\centering
	\includegraphics[scale=0.7]{images/useCaseA}
	\caption{Use case diagram - Bejelentkezett felhasználó}
	\label{abra:useCaseA}
\end{figure}
\pagebreak

Az adminisztrátorok a harmadik kategóriát alkotják a rendszerben, és különleges jogosultságokkal rendelkeznek. Az \textbf{Admin} felhasználóknak joguk van az összes korábban említett funkcionalitáshoz, mint például a jegyek kosárba helyezése, visszajelzések írása és értékelése, valamint a profiljuk kezelése és vásárlási előzményeik megtekintése. Azonban az Admin felhasználóknak további feladatuk is van, nevezetesen a jegyek hitelesítése.

Az Admin felhasználóknak lehetőségük van a jegyek hitelesítésére, amelyhez a QR kódot be kell olvasniuk és meg kell adniuk a PIN kódot. Ez a folyamat biztosítja, hogy csak érvényes jegyek kerüljenek elfogadásra és használatra a rendszerben. Ez a szerepkör manuálisan adható hozzá a rendszerhez, és szükség esetén módosítható.

Fontos megjegyezni, hogy bár az adminisztrátoroknak lehetőségük van jegyek vásárlására, ezt kizárólag a rendszer tesztelésére és karbantartására szolgáló célokra használható. Az adminisztrátorok elsődleges felelőssége a rendszer megfelelő működésének biztosítása és a jegyek hitelesítése (\ref{abra:useCaseAdmin}).

\begin{figure}[!h]
	\centering
	\includegraphics[scale=0.7]{images/useCaseAdmin}
	\caption{Use case diagram - Adminisztrátor}
	\label{abra:useCaseAdmin}
\end{figure}
\pagebreak

  	\item[\textbf{b,}] \textbf{Jegyvásárlás:}

A \textbf{jegyvásárlás} során a felhasználóknak számos funkcionalitást kell biztosítani. Először is, lehetőséget kell adni nekik, hogy kiválaszthassák a jegyeket és azokat kosárba helyezhessék. Emellett fontos, hogy a felhasználók módosíthassák a jegyek mennyiségét vagy akár törölhessék azokat, ha szükséges. A böngészés során pedig lehetővé kell tenni számukra, hogy részletes információkat kapjanak a jegyekről, mint például a helyszín, az időpont vagy az ár. Ez segíti őket a megfelelő döntéshozatalban és a kívánt jegyek megtalálásában.


  	\item[\textbf{c,}] \textbf{Bejelentkezés és regisztráció:}

A \textbf{bejelentkezés} és \textbf{regisztráció} lehetővé teszik a felhasználók számára, hogy saját profiljuk legyen a rendszerben, amellyel vásárlásokat tudnak végrehajtani és nyomon tudják követni azokat.

A \textbf{bejelentkezés} folyamata magában foglalja a felhasználónév és jelszó megadását vagy egy harmadik féltől származó fiók segítségével. A felhasználóknak lehetőségük van megadni a regisztrált felhasználónevüket és a hozzájuk tartozó jelszavukat a bejelentkezéshez. A felhasználói felületnek tartalmaznia kell egy bejelentkezés gombot, amelyre kattintva a felhasználó bejelentkezik a rendszerbe. Ezután a rendszer ellenőrzi az adatok helyességét és hitelesíti a felhasználót, hogy hozzáférjen a felhasználói funkciókhoz.

A \textbf{regisztráció} lehetőséget ad a felhasználóknak arra, hogy új fiókot hozzanak létre a rendszerben. Ehhez a felhasználóknak kitöltetniük kell egy regisztrációs űrlapot, amely tartalmazza az szükséges adatokat, például felhasználónevüket, jelszavukat, e-mail címüket. A felületnek tartalmaznia kell egy regisztrációs gombot, amelyre kattintva a rendszer regisztrálja a felhasználót és létrehozza az új fiókját.
\end{itemize}

\subsection {Nem funkcionális követelmények}

A nem funkcionális követelmények olyan aspektusokra fókuszálnak, amelyek nem közvetlenül a rendszer funkcionalitásához kapcsolódnak, hanem inkább annak működési jellemzőit, tulajdonságait vagy környezeti tényezőit érintik.

\begin{itemize}
  	\item[\textbf{a,}] \textbf{Felhasználói szerepkörök:}

A \textbf{felhasználói szerepkörök} között egyértelmű határvonal kell legyen, hogy melyeknek mihez van jogosultságuk. A felhasználóknak könnyen és zökkenőmentesen kell tudniuk használni az oldalt függetlenül a szerepkörüktől. Fontos továbbá, hogy az egyes szerepkörökhöz tartozó jogosultságok és hozzáférési szintek megbízhatóak és biztonságosak legyenek, hogy a felhasználók csak azokhoz az információkhoz és funkciókhoz férjenek hozzá, amelyek az adott szerepkörhöz kötött.

 	 \item[\textbf{b,}] \textbf{Jegyvásárlás:}

A \textbf{jegyvásárlási} folyamatnak gyorsnak és hatékonynak kell lennie. A rendszernek képesnek kell lennie a nagyobb mennyiségű jegykezelésre és skálázhatónak kell lennie, hogy a jegyvásárlás során ne jelentkezzenek teljesítménybeli problémák. Emellett a jegyvásárlás folyamatának biztonságosnak is kell lennie, és megfelelő védelmi intézkedéseket kell tartalmaznia az adatbiztonság érdekében. Ilyen intézkedések például a titkosított adatátvitel vagy a vásárlói adatok megfelelő védelme.

	 \item[\textbf{c,}] \textbf{Bejelentkezés és regisztráció:}

A \textbf{bejelentkezés és regisztráció} folyamata kapcsán a nem funkcionális követelmények között szerepel, hogy a folyamatnak biztonságosnak kell lennie. A bejelentkezési és regisztrációs folyamatnak megfelelő hitelesítési és azonosítási mechanizmusokat kell tartalmaznia. Emellett a folyamatnak gyorsnak és felhasználóbarátnak kell lennie, hogy a felhasználók könnyedén és zökkenőmentesen tudjanak bejelentkezni vagy regisztrálni. A felhasználóknak egyszerű és intuitív felhasználói felületet kell biztosítani a bejelentkezéshez és regisztrációhoz, hogy könnyen megadhassák szükséges információikat és végrehajthassák az adott folyamatot.
\end{itemize}


\section {Felhasználói követelmények}

\begin{itemize}
  	\item[\textbf{a,}] \textbf{Funkcionalitások:}

A főbb funkcionalitások tervezésénél, amelyekről szó volt a funkcionális rendszer követelmények alfejezet (\ref{rendszerFunkcionális}) keretében, figyelembe vettem, hogy azok a lehető legjobban betöltség a feladatukat, így a felhasználók zökkenőmentesen tudják használni a rendszert. Figyelembe vettem az implementáció során a funkcionalitások teljesítményének, megbízhatóságának és használhatóságának a fontosságát.

	\item[\textbf{b,}] \textbf{Felhasználói felület:}

A rendszer felhasználói felületének fejlesztése első sorban arra irányult, hogy intuitív és könnyen érthető legyen, ezáltal a felhasználók ne tapasztaljanak nehézségeket vagy zavarokat az alkalmazás használata során. Egy felhasználó első benyomása határozza meg a legnagyobb mértékben a rendszerről alkotott véleményét, ezért kellő figyelmet fordítottam arra, hogy az oldal letisztult legyen, elkerülve a félrevezető utasításokat és funkcionalitásokat. A felhasználó az oldal első látogatásakor értesül az oldal céljáról és további információkról a hivatalos oldalak elérésére.

	\item[\textbf{c,}] \textbf{Teljesítmény és megbízhatóság:}

Az alkalmazás teljesítménye több komponensből áll össze, amelyek közé tartoznak a reakcióidő, válaszkészség, stabilitás és rendelkezésre állás. Fontosnak tartottam, hogy az alkalmazás gyorsan és megbízhatóan működjön, hogy a felhasználók ne érezzenek frusztráló lassulásokat vagy hibákat. Persze a hibák teljes mértékű elkerülése majdnem hogy lehetetlen a szoftver rendszerek esetében, mivel az emberi hiba lehetősége mindig jelen van. Arra törekedtem, hogy minimalizálva legyen a hibák előfordulásának lehetősége és megelőző intézkedéseket vezettem be. Ide tartoznak azok a megfelelő hibaüzenet is, amelyek kisebb vagy nagyobb hibák esetén is célravezető módon informálják a felhasználót, hogy ez a hiba miként folyásolja be a rendszer további működését vagy adott esetben mit tud tenni a felhasználó a hiba elkerülése ellen.

	\item[\textbf{d,}] \textbf{Profilkezelés:}

A felhasználók számára fontos szempont, hogy hozzáférjenek a saját adataikhoz egy oldalon és modosítani tudják őket. Erre a célra hoztam létre egy aloldalt, ahol kényelmesen lehet profil képet, felhasználónevet módosítani és a rendelési előzmenyeket visszanézni. Továbbá lehetőségük van, hogy az egyes rendelésekhez új PIN kódot rendeljenek.

	\pagebreak
	\item[\textbf{e,}] \textbf{Kompatibilitás:}

Az alkalmazás elsősorban a desktop számítógépekre lett optimalizálva, viszont egy következő továbbfejlesztési lehetőség lenne, hogy reszponzív legyen, ezáltal mobilon és táblagépen kezelhetőbb. A rendszer kompatibilis a közismert modern böngészők mindegyikével, mint a Microsoft Edge, Google Chrome, Mozilla Firefox, Apple Safari vagy az Opera (\ref{abra:browserLogos}).

\begin{figure}[!h]
	\centering
	\includegraphics[scale=0.2]{images/browserLogos}
	\caption{Web böngészők}
	\label{abra:browserLogos}
\end{figure}
\end{itemize}